\documentclass{clbeamer2024}

\usepackage{minted}

\usepackage{minted}
\setminted{
	breaklines=true,
	frame=single,
	bgcolor=lightgray,
	fontsize=\small,
	escapeinside=||
}

\usepackage{xcolor}
\definecolor{bg}{rgb}{0.95, 0.95, 0.92} % Couleur gris clair

\title{
	\includegraphics[width=0.8cm]{logos/Github.png} \hfill
	Introduction à Git et GitHub \hfill
	\includegraphics[width=1.2cm]{logos/Git-Logo.png}
}
\subtitle{Apprendre les bases de la gestion de versions}
\author{Slimani Mohamed Amine}
\institute{}
\date{\today}

\begin{document}
\setcounter{framenumber}{-1}
\frame{\titlepage}



% Sommaire
\begin{frame}{Sommaire}
	\tableofcontents
\end{frame}

\section{Introduction à Git}
\begin{frame}{Qu'est-ce que Git ?}
	\begin{itemize}
		\item Un système de contrôle de version distribué.
		\item Permet de suivre les modifications, collaborer et gérer les versions.
		\item Essentiel pour les développeurs et les équipes techniques.
	\end{itemize}
\end{frame}

\section{Concepts de base de Git}
\begin{frame}{Concepts de base de Git}
	\begin{itemize}
		\item \textbf{Dépôt (Repository)} : Contient l'historique des modifications.
		\item \textbf{Commit} : Un instantané des modifications.
		\item \textbf{Branches} : Permet de travailler sur des fonctionnalités séparées.
		\item \textbf{Merge} : Fusionner des branches.
	\end{itemize}
\end{frame}

\section{Introduction à GitHub}
\begin{frame}{Qu'est-ce que GitHub ?}
	\begin{itemize}
		\item Une plateforme pour héberger des dépôts Git.
		\item Facilite la collaboration et la gestion de projets.
		\item Fonctionnalités : Issues, Pull Requests, Actions.
	\end{itemize}
\end{frame}

\section{Commandes Git essentielles}

\begin{frame}[fragile]{Commandes Git essentielles}
	\begin{exampleblock}{Exemples de commandes}
		\begin{minted}[style=manni, fontsize=\scriptsize, linenos=true]{bash}
# Initialiser un dépôt
git init
			
# Ajouter des fichiers
git add fichier.txt
			
# Faire un commit
git commit -m "Message de commit"
			
# Ajouter un dépôt distant
git remote add origin https://github.com/utilisateur/mon-depot.git
			
# Vérifier les dépôts distants
git remote -v
			
# Pousser des modifications
git push -u origin main
		\end{minted}
	\end{exampleblock}
\end{frame}


\section{Bonnes pratiques}
\begin{frame}{Bonnes pratiques}
	\begin{itemize}
		\item Utiliser des messages de commit clairs et descriptifs.
		\item Travailler avec des branches pour chaque fonctionnalité.
		\item Utiliser `.gitignore` pour ignorer les fichiers inutiles.
		\item Collaborer via des pull requests.
	\end{itemize}
\end{frame}





\end{document}
