\documentclass{clbeamer2024}

\usepackage{minted}

\usepackage{minted}
\setminted{
	breaklines=true,
	frame=single,
	bgcolor=lightgray,
	fontsize=\small,
	escapeinside=||
}

\usepackage{xcolor}
\definecolor{bg}{rgb}{0.95, 0.95, 0.92} % Couleur gris clair

\title{
	\includegraphics[width=1cm]{logos/cloud.png} \hfill
	Introduction au Cloud Computing
	\includegraphics[width=1cm]{logos/cloud2.png} \hfill
}
\subtitle{Comprendre les bases du cloud et ses avantages}
\author{Slimani Mohamed Amine}
\institute{}
\date{\today}

\begin{document}
	\setcounter{framenumber}{-1}
	\frame{\titlepage}
	
	
	
	% Sommaire
	\begin{frame}{Sommaire}
		\tableofcontents
	\end{frame}
	
	\section{Qu'est-ce que le Cloud Computing ?}
	\begin{frame}{Qu'est-ce que le Cloud Computing ?}
		\begin{itemize}
			\item \textbf{Définition} : Le cloud computing est la fourniture de services informatiques via Internet.
			\item \textbf{Modèles de service} :
			\begin{itemize}
				\item \textbf{IaaS} : Infrastructure as a Service (ex : AWS EC2).
				\item \textbf{PaaS} : Platform as a Service (ex : Google App Engine).
				\item \textbf{SaaS} : Software as a Service (ex : Google Workspace).
			\end{itemize}
			\item \textbf{Avantages} : Flexibilité, scalabilité, réduction des coûts.
		\end{itemize}
	\end{frame}
	
	
	\section{Pourquoi utiliser le Cloud ?}
	\begin{frame}{Pourquoi utiliser le Cloud ?}
		\begin{itemize}
			\item \textbf{Flexibilité} : Accéder aux ressources à la demande.
			\item \textbf{Scalabilité} : Ajuster les ressources en fonction des besoins.
			\item \textbf{Réduction des coûts} : Pas besoin d'investir dans du matériel coûteux.
			\item \textbf{Accès global} : Accéder aux données et applications de n'importe où.
		\end{itemize}
	\end{frame}
	
	
	\section{Types de Cloud}
	\begin{frame}{Types de Cloud}
		\begin{itemize}
			\item \textbf{Cloud Public} : Services fournis par des tiers (ex : AWS, Azure, Google Cloud).
			\item \textbf{Cloud Privé} : Infrastructure dédiée à une seule organisation.
			\item \textbf{Cloud Hybride} : Combinaison de cloud public et privé.
		\end{itemize}
	\end{frame}
	
   \section{Principaux fournisseurs de Cloud}
   \begin{frame}{Principaux fournisseurs de Cloud}
   	\begin{itemize}
   		\item \textbf{Amazon Web Services (AWS)} : Leader du marché avec une large gamme de services.
   		\item \textbf{Microsoft Azure} : Forte intégration avec les produits Microsoft.
   		\item \textbf{Google Cloud Platform (GCP)} : Connu pour ses services de machine learning et de big data.
   	\end{itemize}
   \end{frame}
   
   \section{Services Cloud populaires}
   \begin{frame}{Services Cloud populaires}
   	\begin{itemize}
   		\item \textbf{Calcul} : Machines virtuelles, conteneurs, fonctions serverless.
   		\item \textbf{Stockage} : Stockage objet, stockage bloc, stockage fichier.
   		\item \textbf{Bases de données} : Bases de données relationnelles, NoSQL, data warehouses.
   		\item \textbf{Réseau} : Réseaux virtuels, équilibreurs de charge, CDN.
   	\end{itemize}
   \end{frame}


\section{Exemple de déploiement sur le Cloud}
\begin{frame}{Exemple de déploiement sur le Cloud}
	\begin{enumerate}
		\item Créer une machine virtuelle sur AWS EC2.
		\item Configurer un stockage S3 pour les fichiers.
		\item Déployer une application sur la machine virtuelle.
		\item Configurer un équilibreur de charge pour la haute disponibilité.
	\end{enumerate}
\end{frame}


\section{Bonnes pratiques}
\begin{frame}{Bonnes pratiques}
	\begin{itemize}
		\item \textbf{Sécurité} : Utiliser des groupes de sécurité, des VPN, et des politiques d'accès.
		\item \textbf{Optimisation des coûts} : Surveiller l'utilisation des ressources et utiliser des instances réservées.
		\item \textbf{Gestion des données} : Sauvegarder régulièrement les données et utiliser des services de réplication.
	\end{itemize}
\end{frame}

\section{Outils pour travailler avec le Cloud}
\begin{frame}{Outils pour travailler avec le Cloud}
	\begin{itemize}
		\item \textbf{Terraform} : Outil d'infrastructure as code pour gérer les ressources cloud.
		\item \textbf{Ansible} : Outil d'automatisation pour la configuration et le déploiement.
		\item \textbf{CloudWatch (AWS)} : Service de surveillance pour les ressources AWS.
	\end{itemize}
\end{frame}


\section{Pourquoi c'est important ?}
\begin{frame}{Pourquoi c'est important ?}
	\begin{itemize}
		\item Le cloud computing est essentiel pour les entreprises modernes.
		\item Il permet de stocker, traiter et gérer des données et des applications de manière flexible et scalable.
		\item Comprendre le cloud est crucial pour les développeurs, les administrateurs système, et les professionnels de l'informatique.
	\end{itemize}
\end{frame}


\begin{frame}{Résumé}
	\textbf{Le cloud computing} est une technologie clé pour le stockage, le traitement des données, et le déploiement d'applications.  
	Automatisez pour déployer mieux, plus vite, et avec confiance !
\end{frame}

	
	
\end{document}
