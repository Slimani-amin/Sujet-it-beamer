\documentclass{clbeamer2024}

\usepackage{minted}

\usepackage{minted}
\setminted{
	breaklines=true,
	frame=single,
	bgcolor=lightgray,
	fontsize=\small,
	escapeinside=||
}

\usepackage{xcolor}
\definecolor{bg}{rgb}{0.95, 0.95, 0.92} % Couleur gris clair

\title{
	\includegraphics[width=0.8cm]{logos/Http.png} \hfill
	HTTP/HTTPS \hfill
	\includegraphics[width=0.8cm]{logos/https.png}
}
\subtitle{Comprendre les bases des protocoles web}
\author{Slimani Mohamed Amine}
\institute{}
\date{\today}

\begin{document}
	\setcounter{framenumber}{-1}
	\frame{\titlepage}
	
	
	
	% Sommaire
	\begin{frame}{Sommaire}
		\tableofcontents
	\end{frame}
	

\section{Qu'est-ce que HTTP ?}
\begin{frame}{Qu'est-ce que HTTP ?}
	\begin{itemize}
		\item \textbf{Définition} : Un protocole de communication utilisé pour transférer des données sur le web.
		\item \textbf{Fonctionnement} : Requête-réponse entre un client (navigateur) et un serveur.
		\item \textbf{Exemple} : Lorsque vous visitez un site web, votre navigateur envoie une requête HTTP au serveur, qui répond avec le contenu de la page.
	\end{itemize}
\end{frame}

\section{Qu'est-ce que HTTPS ?}
\begin{frame}{Qu'est-ce que HTTPS ?}
	\begin{itemize}
		\item \textbf{Définition} : Une version sécurisée de HTTP qui utilise le chiffrement SSL/TLS.
		\item \textbf{Pourquoi HTTPS ?} : Protège les données contre les interceptions et les attaques.
		\item \textbf{Exemple} : Les sites de commerce électronique utilisent HTTPS pour sécuriser les transactions.
	\end{itemize}
\end{frame}

\section{Différences entre HTTP et HTTPS}
\begin{frame}{Différences entre HTTP et HTTPS}
	\begin{itemize}
		\item \textbf{Chiffrement} : HTTPS chiffre les données, HTTP non.
		\item \textbf{Port} : HTTP utilise le port 80, HTTPS utilise le port 443.
		\item \textbf{Certificats} : HTTPS nécessite un certificat SSL/TLS.
	\end{itemize}
\end{frame}


\section{Requête HTTP/HTTPS}
\begin{frame}{Requête HTTP/HTTPS}
	\begin{itemize}
		\item \textbf{Structure d'une requête HTTP} :
		\begin{itemize}
			\item Méthode (GET, POST, etc.)
			\item En-têtes (Headers)
			\item Corps (Body)
		\end{itemize}
		\item \textbf{Structure d'une réponse HTTP} :
		\begin{itemize}
			\item Code de statut (200, 404, etc.)
			\item En-têtes (Headers)
			\item Corps (Body)
		\end{itemize}
	\end{itemize}
\end{frame}

\section{Bonnes pratiques}
\begin{frame}{Bonnes pratiques}
	\begin{itemize}
		\item \textbf{Toujours utiliser HTTPS} : Pour protéger les données des utilisateurs.
		\item \textbf{Configurer correctement les certificats SSL/TLS} : Utiliser des certificats valides et à jour.
		\item \textbf{Utiliser HSTS (HTTP Strict Transport Security)} : Pour forcer l'utilisation de HTTPS.
	\end{itemize}
\end{frame}

\section{Outils pour tester HTTP/HTTPS}
\begin{frame}{Outils pour tester HTTP/HTTPS}
	\begin{itemize}
		\item \textbf{curl} : Un outil en ligne de commande pour envoyer des requêtes HTTP/HTTPS.
		\item \textbf{Postman} : Un outil graphique pour tester les API HTTP/HTTPS.
		\item \textbf{SSL Labs} : Un site web pour vérifier la configuration SSL/TLS d'un serveur.
	\end{itemize}
	\end{frame}
	
	
	\section{Exemple de requête avec \texttt{curl}}
	\begin{frame}[fragile]{Exemple de requête avec \texttt{curl}}
		\begin{exampleblock}{Commandes \texttt{curl}}
			\begin{minted}[fontsize=\scriptsize]{bash}
# Requête HTTP
curl -v http://example.com
				
# Requête HTTPS
curl -v https://example.com
			\end{minted}
		\end{exampleblock}
	\end{frame}
	
	\section{Pourquoi c'est important ?}
	\begin{frame}{Pourquoi c'est important ?}
		\begin{itemize}
			\item HTTP/HTTPS sont les protocoles de communication les plus utilisés sur le web.
			\item HTTPS est essentiel pour protéger les données des utilisateurs.
			\item Comprendre leur fonctionnement est crucial pour développer des applications web sécurisées.
		\end{itemize}
	\end{frame}
	
	
\end{document}
